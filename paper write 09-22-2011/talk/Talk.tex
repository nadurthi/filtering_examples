%%%%%%%%%%%%%%%%%%%%%%%%%%%%%%%%%%%%%%%%%%%%%%%%%%%%%%%%%%%%%%%%%%%%%%%
%
%   Presentation of Beamer UNL Theme
%   Beamer Presentation by Chris Bourke
%
%%%%%%%%%%%%%%%%%%%%%%%%%%%%%%%%%%%%%%%%%%%%%%%%%%%%%%%%%%%%%%%%%%%%%%%%
%\documentclass{article}
%\usepackage{beamerarticle}
\documentclass{beamer}
%\documentclass[brown]{beamer}

\mode<presentation> {

\usetheme{waqar}%{waqar}
%\setbeamertemplate{background canvas}[vertical shading][bottom=red!10,top=blue!10]
%\usecolortheme{rose}
\usecolortheme{wcolor}
\definecolor{ublue}{rgb}{000,051,153}
\definecolor{ugray}{rgb}{102,102,102}
%\usecolortheme[named=wgreen]{structure}
\setbeamercovered{transparent}
\usefonttheme{serif}
\usefonttheme{structuresmallcapsserif}
%\usefonttheme{structuresmallcaps}
\usefonttheme{professionalfonts}
} \definecolor{maroon}{RGB}{149,0,0} \setbeamercolor{alerted
text}{fg=maroon}

\usepackage[english]{babel}
\usepackage[latin1]{inputenc}
\usepackage{times}
\usepackage[T1]{fontenc}
\usepackage[slantedGreek]{mathptmx}
\usepackage{amsmath,amssymb,bm}
\usepackage{pgf,subfigure,wrapfig,hyperref}

\graphicspath{{/Users/psingla/Documents/PuneetDocs/Publications/Conferences/ACC/ACC10/stochForcingLin/Figures/}}
\newcommand{\incfig}{\includegraphics}

\title[Adaptive GSF]{Certain Thoughts on Uncertainty Analysis for Dynamical Systems}
\author[\color{white}{P.~Singla (\href{mailto:psingla@buffalo.edu}{psingla(@)buffalo.edu})}]{\normalsize Puneet Singla \\\textit{Assistant Professor\\Department of Mechanical \& Aerospace Engineering\\ University at Buffalo, Buffalo, NY-14260}}
\institute[\color{white}{University at Buffalo}]{\normalsize\alert{Probabilistic Analysis of Volcanic Hazards:
Current Methodologies and Vision for Future Efforts}\\\vspace{0.06in}{\small\color{maroon}{\textit{A Workshop at the University at Buffalo\\May~16--May~19, 2011}}}}
\date[\href{http://lairs.eng.buffalo.edu}{lairs.eng.buffalo.edu}]{\footnotesize\color{blue}{Collaborators: M.~Bursik, M.~Jones, A.~Patra, E.~B.~Pitman, P.~Scott, T.~Singh\\H.~Bjornsson, S.~Carn, K.~Dean, M.~Pavolonis, M.~Ripepe, P.~Webley}\\\vspace{.2in}\insertlogo{\incfig[width=1.5in]{ub.pdf}} }
\newtheorem{distheorem}{Theorem}

\newcommand{\re}{\mathbb R}
\newcommand{\Ex}[1]{\mathbf{E}[#1]}
\newcommand{\Tr}{\mathbf{Tr}}
\newcommand{\etal}{\textit{et al. }}

\newcommand{\eq}[1]{Eq.~(\ref{#1})}
\newcommand{\Fig}[1]{Fig.~\ref{#1}}

\renewcommand{\v}[1]{\ensuremath{\mathbf{#1}}}
\newcommand{\vo}[1]{\ensuremath{\boldsymbol{#1}}}
\newcommand{\pderiv}[2]{\frac{\partial #1}{\partial #2}}

\newcommand{\Th}{\boldsymbol{\Theta}}
\newcommand{\A}{\mathbf{A}}
\newcommand{\B}{\mathbf{B}}
\newcommand{\G}{\mathbf{G}}
\newcommand{\Q}{\mathbf{Q}}
\newcommand{\z}{\mathbf{z}}
\renewcommand{\L}{\mathbf{L}}
\newcommand{\M}{\mathbf{M}}
\newcommand{\w}{\mathbf{w}}

\renewcommand{\th}{\theta}
\newcommand{\eps}{\boldsymbol{\epsilon}}
\newcommand{\me}{\boldsymbol{\mu}}
\newcommand{\cov}{\boldsymbol{\Sigma}}
\newcommand{\xib}{\boldsymbol{\xi}}
\newcommand{\om}{\boldsymbol{\Omega}}



\newcommand{\Ps}{\boldsymbol{\Psi}}
\newcommand{\Al}{\boldsymbol{\alpha}}
\newcommand{\x}{\mathbf{x}}
\renewcommand{\u}{\mathbf{u}}
\newcommand{\et}{\boldsymbol{\eta}}


\begin{document}

%%%%%%
\begin{frame}[plain]
\begin{center}
  \titlepage
  \end{center}
\end{frame}
%\begin{frame}\frametitle{Contents}
%\tableofcontents
%\end{frame}
\section{introduction}
\begin{frame}\frametitle{\color{black}{Introduction}}\framesubtitle{\color{white}{Problem Statement}}
\begin{itemize}%[<+->]
\item<1-> Stochastic System: $\dot{\x} = \A(\Th)\x + \B(\Th)\u+ \G(\Th)\et,\ \x(t_0) =\me_0$.
\begin{itemize}
\item \alert{Source of Uncertainties:} system parameters, initial conditions, input to the system, modeling error.
\end{itemize}
\item Robust modeling of the propagation of these uncertainties is important to \alert{accurately quantify the uncertainty} in the solution at any future time.
\end{itemize}
\begin{figure}\vspace{-0.1in}
\begin{center}
\includegraphics[width=2.3in]{/Users/psingla/Documents/PuneetDocs/Publications/Conferences/ACC/ACC10/stochForcingLin/Figures/pdf_tran.pdf}\vspace{-0.15in}
\caption{State and pdf transition.}
\end{center}
\end{figure}
\end{frame}

\begin{frame}\frametitle{\color{black}{Introduction}}\framesubtitle{\color{white}{Problem Statement}}
%\transblindshorizontal
\begin{block}<1->{Linear systems with known parameter: $\dot{\x} = \A(\Th_0)\x + \B(\Th_0)\u+ \G(\Th_0)\et$}
\begin{itemize}
\item State pdf is Gaussian assuming initial condition and input uncertainty to be Gaussian in nature.
\item \alert{Drawback:} appending unknown parameters as additional state variables lead to non-linear system with non-gaussian pdfs.
\end{itemize}
\end{block}

\begin{block}<2->{Linear systems with unknown parameter: $\dot{\x} = \A(\Th)\x + \B(\Th)\u+ \G(\Th)\et$}
\begin{itemize}
\item Generalized Polynomial Chaos (gPC): propagate time-invariant parametric uncertainty through an otherwise deterministic system of equations.

\item \alert{Drawback:} Using gPC series expansion for the time-varying stochastic forcing terms is computationally expensive.
\end{itemize}
\end{block}

\end{frame}

\begin{frame}\frametitle{\color{black}{Introduction}}\framesubtitle{\color{white}{Motivation}}

\begin{itemize}
\item Main challenge: characterizing the uncertainty in the system states due to {\color{blue}{\textit{both parametric and temporal stochastic uncertainties} simultaneously}}
\begin{itemize}
\item \alert{Approximate Solution to exact problem:} Multiple-model estimation method, Monte Carlo (MC) methods.
\item \alert{Exact solution to approximate problem:} Gaussian closure, Equivalent Linearization, and Stochastic Averaging.
\end{itemize}
\item<2-> {\color{blue}{Main Objective:}} \textit{develop analytical means to accurately characterize the state pdf of a linear system subject to initial condition uncertainty, white noise excitation and possibly non-Gaussian parametric uncertainty.}
\begin{itemize}
\item<2-> {\color{blue}{Uncertainty Marriage:}} \alert{handshake of Kalman filter with gPC.}
\end{itemize}

\end{itemize}
\end{frame}

\section{Proposed Approach}
\begin{frame}\frametitle{\color{black}{Proposed Approach}}\framesubtitle{\color{white}{$\dot{\x} = \A(\Th)\x + \B(\Th)\u+ \G(\Th)\et$}}

\begin{block}{Method 1: Conditioning first on uncertain parameters}
\begin{itemize}
\item The conditional state pdf $p(\x|\Th)$ is a normal distribution with mean $\me$ and covariance $\cov$, i.e. $p(\x|\Th)=\mathcal{N}(\x;\me,\cov)$.
\begin{align}\label{meanG}
\dot{\me} &=\A(\Th)\me + \B\u\\
\dot{\cov} &= \A(\Th)\cov+\cov\A^T(\Th)+\G(\Th) \Q \G^T(\Th) \label{covG}
\end{align}
\item These conditional moment propagation equations are \alert{exact and depend only} on the initial moments and the model parameters.
\item The complete distribution of the original state vector $\x$ is:\vspace{-0.15in}
\begin{align}\footnotesize
p(t,\x) &= \int\limits_{\om}{p(t,\x|\Th(\xib))p(\xib)d\xib}\\
&=\int\limits_{\om}\mathcal{N}(t,\x;\me(t,\Th),\cov(t,\Th))p(\xib)d\xib \label{meth2pi}
\end{align}
\end{itemize}
\end{block}
%To avoid Monte Carlo sampling, we make use of gPC to obtain the distribution for $\me$ and covariance $\cov$ in terms of the random variable $\xib$.
\end{frame}


\begin{frame}\frametitle{\color{black}{Proposed Approach}}\framesubtitle{\color{white}{$\dot{\x} = \A(\Th)\x + \B(\Th)\u+ \G(\Th)\et$}}
\begin{block}<1>{Polynomial Chaos}
\begin{itemize}
\item Polynomial chaos is a term originated by \textit{Norbert Wiener in} $1938$, to describe the members of the {\color{blue}{span of Hermite polynomial functionals of a Gaussian process.}}
\item According to the Cameron-Martin Theorem, the Fourier-Hermite polynomial chaos expansion converges, in the $L^2$ sense, to \textit{any arbitrary process with finite variance} (which applies to most physical processes).
\item This has been generalized by Xiu et al. (2002) to \alert{efficiently use the orthogonal polynomials} from the Askey-scheme to model various probability distributions.
\end{itemize}
\end{block}
\end{frame}

\begin{frame}\frametitle{\color{black}{Proposed Approach}}\framesubtitle{\color{white}{$\dot{\x} = \A(\Th)\x + \B(\Th)\u+ \G(\Th)\et$}}

\begin{block}{Generalized Polynomial Chaos: Basic Goal}
\begin{itemize}
\item Approximate the stochastic system states in terms of \alert{a finite-dimensional series expansion} in the infinite-dimensional stochastic space.
\item The unknown coefficients are determined by {\color{blue}{minimizing an appropriate norm of the residual.}}
\item The basis can be chosen for a given pdf, to represent the random variable with the fewest number of terms.
\begin{itemize}
\item For example, Hermite polynomials for Gaussian r.v. and Legendre polynomials for uniform r.v.
\end{itemize}
\item \alert{The completeness of the space} allows for the accurate representation of any random variable, with a given probability density function (pdf), by a \textit{suitable projection on the space spanned by a carefully selected basis.}
\end{itemize}
\end{block}
 \end{frame}


\begin{frame}\frametitle{\color{black}{Proposed Approach}}\framesubtitle{\color{white}{$\dot{\x} = \A(\Th)\x + \B(\Th)\u+ \G(\Th)\et$}}

\begin{block}{Method 1: $p(t,\x)=\int\limits_{\om}\mathcal{N}(t,\x;\me(t,\Th),\cov(t,\Th))p(\xib)d\xib $}
\begin{itemize}
\item<1-> $p(\x|\Th)=\mathcal{N}(\x;\me,\cov)$
\begin{align}\label{meanG}
\dot{\me} &=\A(\Th)\me + \B\u\\
\dot{\cov} &= \A(\Th)\cov+\cov\A^T(\Th)+\G(\Th) \Q \G^T(\Th)
\end{align}
\item Combining the state mean and covariance terms into a new state vector $\z \in \re^N$ where $N = n(n+3)/2$ leads to
\begin{equation}
\dot{\z}= \L(\Th)\z+\M(\Th)\w \label{sysdet2}
\end{equation}
\end{itemize}
\end{block}
\end{frame}

\begin{frame}\frametitle{\color{black}{Proposed Approach}}\framesubtitle{\color{white}{$\dot{\x} = \A(\Th)\x + \B(\Th)\u+ \G(\Th)\et$}}
\vspace{-0.1in}
\begin{block}{Method 1: $\dot{\z}= \L(\Th)\z+\M(\Th)\w $}
\begin{itemize}
\item According to gPC
\small{\begin{eqnarray}
z_{i}(t,\xib) &= \sum\limits_{r=0}^P {z_i}_r(t)\phi_r(\xib) =& \z_i^T(t)\Phi(\xib) \label{states}\\
\Th_j(\xib) &= \sum\limits_{r=0}^P {\th_j}_r\phi_r(\xib) =& \th_j^T\Phi(\xib) \label{params}\\
L_{ij}(\Th) &= \sum\limits_{r=0}^P {L_{ij}}_r\phi_r(\xib) =& \v{l}_{ij}^T\Phi(\xib) \label{paramsA}\\
M_{ij}(\Th) &= \sum\limits_{r=0}^P {M_{ij}}_r\phi_r(\xib) =& \v{m}_{ij}^T\Phi(\xib) \label{paramsB}
\end{eqnarray}}
\end{itemize}
\end{block}
\small{${L_{ij}}_r = \frac{\left\langle L_{ij}(\Th(\xib)),\phi_r(\xib)\right\rangle}{\left\langle \phi_r(\xib),\phi_r(\xib)\right\rangle}$
${M_{ij}}_r = \frac{\left\langle M_{ij}(\Th(\xib)),\phi_r(\xib)\right\rangle}{\left\langle \phi_r(\xib),\phi_r(\xib)\right\rangle}$\\\vspace{0.2in}


\alert{where $\left\langle u(\xib),v(\xib)\right\rangle = \int_{\om}{u(\xib)v(\xib)p(\xib)d\xib}$}}
\end{frame}

\begin{frame}\frametitle{\color{black}{Proposed Approach}}\framesubtitle{\color{white}{$\dot{\x} = \A(\Th)\x + \B(\Th)\u+ \G(\Th)\et$}}

\begin{block}{Method 1: $\dot{\z}= \L(\Th)\z+\M(\Th)\w $}
\begin{itemize}
\item Substitution of the approximate expressions for $\x$ and $\Th$ in moment equations leads to:
\begin{align}\notag
e_i(\xib) = &\dot{\z}_i^T\Phi(\xib)-\sum_{j=1}^N{\v{l}_{ij}^T\Phi(\xib)\z_j^T(t)\Phi(\xib)}-\sum_{j=1}^q{\v{m}_{ij}^T\Phi(\xib)w_j(t)}
\end{align}
\item Galerkin projection leads to following deterministic equations:
\begin{align}
&\left\langle e_i(\xib),\phi_r(\xib)\right\rangle = 0 \mbox{ for $i = 1,\ldots,N$ and $r = 0,\ldots,P$} \notag\\
&\alert{\dot{\v{c}} = \A_p\v{c}+\B_p\w}
\end{align}
where, $\v{c}(t) = [\z_1^T(t),\z_2^T(t),\ldots,\z_N^T(t)]^T \in \re^{N(P+1)}$ is a vector of the gPC coefficients.
\end{itemize}
\end{block}
\end{frame}

\begin{frame}\frametitle{\color{black}{Proposed Approach}}\framesubtitle{\color{white}{$\dot{\x} = \A(\Th)\x + \B(\Th)\u+ \G(\Th)\et$}}

\begin{block}{Method 1: $p(t,\x)=\int\limits_{\om}\mathcal{N}(t,\x;\me(t,\Th),\cov(t,\Th))p(\xib)d\xib $}
\begin{itemize}
\item The pdf of state vector $\x$ can be computed as follows:\vspace{-0.15in}
\small{\begin{align}
p(\x)&=\int\limits_{\om}\mathcal{N}\left(t,\x;\sum_{r=0}^P {z_i}_r(t)\phi_r(\xib)\right)p(\xib)d\xib \label{pdfeq1}
\end{align}}\vspace{-0.2in}
\item \alert{Quadrature integration} scheme can be used to evaluate the aforementioned integrals.
\begin{itemize}
\item Polynomial Chaos Quadrature (PCQ) {\color{blue}{[Dalbey et al., 2008]}}
\end{itemize}
\item<2-> The first two moments of the actual state vector $\x$ can be estimated analytically, as follows:\vspace{-0.2in}
\small{\begin{align}
\Ex{x_i(t)} &= \int\limits_{\om}{\Ex{\mathcal{N}(t,\x;\sum_{r=0}^P {z_i}_r(t)\phi_r(\xib))} p(\xib)d\xib}= {\mu_i}_0(t)\\
\Ex{x_i^2(t)} &= \int\limits_{\om}{\Ex{x_i^2(t)|\xib} p(\xib)d\xib}= \sum_{r=0}^P {\mu_i}_r^2(t)\langle\phi_r^2\rangle + \Sigma_{ii_0}(t)\vspace{-0.1in}
\end{align}}\vspace{-0.1in}
\end{itemize}
\end{block}
\end{frame}

\begin{frame}\frametitle{\color{black}{Proposed Approach}}\framesubtitle{\color{white}{$\dot{\x} = \A(\Th)\x + \B(\Th)\u+ \G(\Th)\et$}}
\transboxin

\begin{figure}\vspace{-0.22in}\centering
\subfigure[\scriptsize{Method 1: Conditioning first on uncertain parameters [Konda et al., 2011]}]{\includegraphics[width=3.8in]{uncertprop1lq.pdf}\label{fig:meth1}\vspace{-0.2in}}
\end{figure}
\end{frame}
\begin{frame}\frametitle{\color{black}{Proposed Approach}}\framesubtitle{\color{white}{$\dot{\x} = \A(\Th)\x + \B(\Th)\u+ \G(\Th)\et$}}
\begin{figure}\vspace{-0.22in}\centering
\subfigure[\scriptsize{Method 2: Conditioning first on Gaussian stochastic forcing [Konda et al., 2011]}]{\includegraphics[width=3.8in]{uncertprop2lq.pdf}\label{fig:meth2}}
\end{figure}
\end{frame}

\begin{frame}\frametitle{\color{black}{Proposed Approach}}\framesubtitle{\color{white}{Advantages}}
\begin{itemize}
\item One can compute the sensitivity of mean and covariance of conditional pdf $p(t,\x|\Th) = \mathcal{N}(t,\x;\z(\xib))$ with respect to unknown parameter vector $\Th(\xib)$.
\item \alert{Computational Cost:}
\begin{itemize}
\item Method $1$: $n(n+3)(P+1)/2$ simultaneous deterministic ODEs.
\item Method $2$: $n(P+1)\left[n(P+1)+3\right]/2$ simultaneous deterministic ODEs.
\item SMC: $nN$ differential equations, $N$ being number of Monte Carlo samples.
\end{itemize}
\end{itemize}
\end{frame}
%%%%%%%%%%%%%%%%%%%%%%%%%%%%%%%%%%%%%%%%%%%%%%%%%%%%%%%%%%%%%%%%%%%%%5

\section{Numerical Results}
%\begin{frame}\frametitle{Numerical Experiment}\framesubtitle{\color{white}{Spring Mass Damper System}}
%\begin{itemize}
%\item \alert{$m\ddot{x}+c\dot{x}+kx = u $}
%\begin{itemize}
%\item The spring stiffness $k$ is uncertain and assumed to be {\color{blue}{uniformly distributed}} between $1.5$ and $2.5$.
%\item Input $u$ is assumed to zero mean Gaussian white noise with std. deviation of $2$.
%\item True uncertainty evolution is assumed to be given by \alert{$10,000$ Monte Carlo runs.}
%\item $7^{th}$ order gPC expansion is used to characterize the effect of parametric uncertainty.
%\end{itemize}
%\end{itemize}
%
%
%\begin{figure}[h]\vspace{-0.2in}
%\setlength{\unitlength}{8.0mm}
%\subfigure[Mass-spring-damper system]{
%\begin{picture}(11,5)(1.0,-1.5)
%\thicklines \put(8,0){\framebox(2,2){$m$}} %\put(0,0){\framebox(2,2){$m_1$}}
%\put(2,1.5){\line(1,0){1}} \put(3,1.5){\line(1,1){0.5}} \put(3.5,2.0){\line(1,-1){1}}
%\put(4.5,1.0){\line(1,1){1}} \put(5.5,2.0){\line(1,-1){1}} \put(6.5,1.0){\line(1,1){0.5}}
%\put(7,1.5){\line(1,0){1}} \put(5,1.9){\makebox(0,0)[t]{$k$}} \thinlines
%\put(2,0.5){\line(1,0){3}} \put(5,0.0){\line(0,1){1}} \put(6,0.0){\line(0,1){1}}
%\put(5.5,-0.2){\makebox(0,0)[t]{$c$}} \put(2,-0.5){\line(0,1){3}}
%\put(2,-0.2){\line(-1,-1){0.5}} \put(2,0.1){\line(-1,-1){0.5}}
%\put(2,0.4){\line(-1,-1){0.5}} \put(2,0.7){\line(-1,-1){0.5}}
%\put(2,1.0){\line(-1,-1){0.5}} \put(2,1.3){\line(-1,-1){0.5}}
%\put(2,1.6){\line(-1,-1){0.5}} \put(2,1.9){\line(-1,-1){0.5}}
%\put(2,2.2){\line(-1,-1){0.5}} \put(2,2.5){\line(-1,-1){0.5}}
% \put(6,0.5){\line(1,0){2}}
%\put(4.9,0.0){\line(1,0){1.1}} \put(4.9,1.0){\line(1,0){1.1}}
% \put(9,0){\line(0,-1){0.75}} \put(9,-0.5){\vector(1,0){1}}
%\put(9.5,-0.6){\makebox(0,0)[t]{\small$x$}} %\put(9,2){\line(0,1){0.75}}
%\put(9,2.8){\line(0,-1){0.75}}
%\put(9.0,2.5){\vector(1,0){1}} \put(9.5,2.6){\makebox(0,0)[b]{\small$u$}}
%\end{picture}
%\label{fig:smd}}
%%\subfigure[Evolution of nominal states]{\includegraphics[scale=0.5]{smdnomstates.pdf}\label{fig:nomstates}}
%%\caption{Spring-Mass system} \label{fig:example1}
%\end{figure}
%
%\end{frame}


%\begin{frame}\frametitle{Preliminary Results}\framesubtitle{\color{white}{$m\ddot{x}+c\dot{x}+kx = u,\ k\in\mathcal{U}(1.5,2.5),\ u(t)\in\mathcal{N}(0,2) $}}
%\begin{figure}
%\subfigure[Evolution of nominal states]{\includegraphics[scale=0.5]{smdnomstates.pdf}\label{fig:nomstates}}
%\end{figure}
%\end{frame}

%\begin{frame}\frametitle{Numerical Experiment}\framesubtitle{\color{white}{$m\ddot{x}+c\dot{x}+kx = u,\ k\in\mathcal{U}(1.5,2.5),\ u(t)\in\mathcal{N}(0,2) $}}
%\begin{figure}\vspace{-0.3in}
%\centering
%\subfigure[\footnotesize{Propagation of mean}]{\includegraphics[width=1.7in]{smd_cm1.pdf}\label{fig:smdstmean}}
%\subfigure[\footnotesize{Propagation of variance}]{\includegraphics[width=1.7in]{smd_cm2.pdf}\label{fig:smdstvar}}\vspace{-0.15in}
%\subfigure[\footnotesize{Propagation of 3rd central moment}] {\includegraphics[width=1.7in]{smd_cm3.pdf}\label{fig:smdstcm3}}
%\end{figure}
%\end{frame}
%
%\begin{frame}\frametitle{Numerical Experiment}\framesubtitle{\color{white}{$m\ddot{x}+c\dot{x}+kx = u,\ k\in\mathcal{U}(1.5,2.5),\ u(t)\in\mathcal{N}(0,2) $}}
%\begin{center}
%\begin{figure}
%\includegraphics[width=4.6in]{smd_hist3tpdf10.pdf}
%\caption{Histograms of the states at t = 10s}\label{fig:smdhist}
%\end{figure}
%\end{center}
%\end{frame}

\begin{frame}\frametitle{Numerical Experiment}\framesubtitle{\color{white}{Hovering Helicopter model}}
%\begin{wrapfigure}{r}{1.2in}\vspace{-0.25in}
%%\subfigure[Hovering helicopter\cite{dtenne:04}]{
%\includegraphics[width=1.5in]{bryheli.pdf}\label{fig:bryheli}\vspace{-0.2in}
%%\subfigure[Evolution of nominal states]{\includegraphics[width=2in]{brynomstates.pdf}\label{fig:helinomst}}
%\end{wrapfigure}
\begin{itemize}\vspace{-0.1in}
\item \alert{$\dot{\x} = A\x+B\v{u}+B_w\v{u}_w, $} $\x=\left\{u_h,q_h,\theta_h,y \right\}^T$

\small{$A = \begin{bmatrix} p_1 & p_2 & -g & 0\\ p_3 & p_4 & 0 & 0\\0 & 1 & 0 & 0\\1 & 0 & 0 & 0 \end{bmatrix},$ $B = \begin{bmatrix} p_5\\ p_6\\0\\0\end{bmatrix},$
$B_w = \begin{bmatrix} -p_1\\ -p_3\\0\\0\end{bmatrix}$}

\item \small{Four aerodynamic parameters are assumed to {\color{blue}{uniformly distributed}} with following bounds:}
\footnotesize{$$
\v{p}_{lb} = \begin{bmatrix} -0.049, 0.001, 0.126, -3.354  \end{bmatrix}^T,\  \v{p}_{ub} = \begin{bmatrix} -0.003, 0.025, 2.394, -0.177  \end{bmatrix}^T
$$}
\item \small{{\color{blue}{Modeling errors due to unsteady flow, $u_w$}} are assumed to be \alert{zero mean Gaussian white noise} with a variance of $\sigma_w^2=18 (ft/s)^2$.}
\end{itemize}
\begin{figure}\vspace{-0.2in}
\subfigure[Hovering helicopter]{
\includegraphics[width=1.5in]{bryheli.pdf}\label{fig:bryheli}}
\subfigure[Nominal states]{\includegraphics[width=1.5in]{brynomstates.pdf}\label{fig:helinomst}}
\end{figure}
\end{frame}

\begin{frame}\frametitle{Numerical Experiment}\framesubtitle{\color{white}{Hovering Helicopter model}}\vspace{-0.2in}
\begin{figure}[ht]
\centering
\subfigure[\tiny{Propagation of mean}]{\includegraphics[width=2.1in]{bryheli_1e5cm1.pdf}\label{fig:helistmean}}
\subfigure[\tiny{Propagation of variance}]{\includegraphics[width=2.1in]{bryheli_1e5cm2.pdf}\label{fig:helistvar}}\vspace{-0.05in}
\subfigure[\tiny{Propagation of 3rd central moment}] {\includegraphics[width=2.1in]{bryheli_1e5cm3.pdf}\label{fig:helistcm3}}
\subfigure[\tiny{3rd central moment with 10000 Monte Carlo runs}] {\includegraphics[width=2.1in]{bryheli_cm3.pdf}\label{fig:helistcm3less}}\vspace{-0.2in}
\caption{\scriptsize Propagation of moments of the states [Konda et al., 2011]}\label{fig:helistmoments}
\end{figure}
\end{frame}

\begin{frame}\frametitle{Numerical Experiment}\framesubtitle{\color{white}{Hovering Helicopter model}}\vspace{-0.2in}
%\transblindshorizontal
\begin{center}
\begin{figure}
%\centering
\includegraphics[width=4.6in]{bryheli_1e5hist3tpdf5.pdf}
\caption{\scriptsize Histograms of the states [Konda et al., 2011]}\label{fig:helihist}
\end{figure}
\end{center}
\end{frame}

\begin{frame}\frametitle{Preliminary Results}\framesubtitle{\color{white}{Iceland Volcano (Eyjafjallaj\"{o}kull) Eruption}}
\begin{itemize}
\item<1-> The {\color{blue}{\texttt{BENT} integral eruption column model}} was used to produce eruption column parameters (mass loading, column height, grain size distribution) given \textit{a specific atmospheric sounding and source conditions. }
\begin{itemize}
\item<1->  \texttt{BENT} takes into consideration atmospheric (wind) conditions as given by atmospheric sounding data. 
\item<1-> Plume rise height is given as \alert{\textit{a function of volcanic source and environmental conditions. }}
\end{itemize}
\item<2-> The {\color{blue}{\texttt{PUFF} Lagrangian model}} was used to propagate ash parcels in \textit{a given wind field (NCEP Reanalysis). }
\begin{itemize}
\item<2-> \texttt{PUFF} takes into account dry deposition as well as dispersion and advection.  
\end{itemize}
\item<3-> Polynomial chaos quadrature (PCQ) was used to select sample points and weights in the uncertain input space of \alert{vent radius, vent velocity, mean particle size and particle size variance. }
\end{itemize}
\end{frame}

\begin{frame}\frametitle{Preliminary Results}\framesubtitle{\color{white}{Iceland Volcano (Eyjafjallaj\"{o}kull) Eruption}}
\vspace{-0.2in}
\begin{table}[htb]\footnotesize
\begin{center}\footnotesize
\caption{Eruption source parameters based on observations of
  Eyjafjallaj\"{o}kull volcano and information from other similar
  eruptions.}\label{tab1}
\begin{tabular}{|p{0.6in}|p{0.6in}|p{0.6in}|p{1.5in}|}
\hline \hline
Parameter & Value range & PDF & Comment \\
\hline \hline
Vent radius, $b_0$, m & 65-150 & Uniform & Measured from radar image of summit vents\\
\hline
Vent velocity, $w_0$, m/s & Range: 45-124 & Uniform & M. Ripepe, Geneva, Switzerland, 2010, presentation \\
\hline
Mean grain size, $Md_\varphi$ & 2 boxcars: 1.5-2 and 3-5 & Multi-Modal Uniform & Woods and Bursik (1991), Table 1, vulcanian
and phreatoplinian. A. Hoskuldsson, Iceland meeting 2010, presentation\\%, `vulcanian with unusual production of  fine ash.' \\
\hline
$\sigma_\varphi$ & $1.9 \pm 0.6$ & Uniform& Woods and Bursik (1991), Table 1, vulcanian
and phreatoplinian. \\
\hline \hline
\end{tabular}
\end{center}
\end{table}
\end{frame}

\begin{frame}\frametitle{Preliminary Results}\framesubtitle{\color{white}{Iceland Volcano (Eyjafjallaj\"{o}kull) Eruption}}
\vspace{-0.1in}
\begin{figure}
\centering
\only<1>{\includegraphics[width=3in]{../figures/conver}%\vspace{-0.15in}
\caption{Concentration (52N 13.5E) vs. Number of \texttt{PUFF} Particles}}
\only<2>{\includegraphics[width=3in]{../figures/timing}\vspace{-0.15in}}
\end{figure}
\only<2>{\begin{table}[htb]
\begin{center}\tiny
\begin{tabular}{|p{0.15in}|p{0.42in}|p{0.42in}|p{0.42in}|p{0.42in}|p{0.42in}|p{0.42in}|p{0.42in}|}
\hline\hline
%\only<1>{nAsh &$10^5$ & $5\times10^5$ & $10^6$ & $2\times10^6$ & $4\times10^6$ & $8\times10^6$ & $10^7$   \\\hline\hline
%Conc. & $7.40\times 10^{-5}$ & $1.17\times 10^{-4}$ & $1.07\times 10^{-4}$ & $1.12\times 10^{-4}$ & $1.09\times 10^{-4}$ & $1.15\times 10^{-4}$ & $1.10\times 10^{-4}$\\\hline\hline}
nAsh &$10^5$ & $5\times10^5$ & $10^6$ & $2\times10^6$ & \color{blue}{$4\times10^6$} & $8\times10^6$ & $10^7$   \\\hline\hline
Conc. & $7.40\times 10^{-5}$ & $1.17\times 10^{-4}$ & $1.07\times 10^{-4}$ & $1.12\times 10^{-4}$ & \color{blue}{$1.09\times 10^{-4}$} & $1.15\times 10^{-4}$ & $1.10\times 10^{-4}$\\\hline\hline
\end{tabular}
\end{center}
\end{table}}
\end{frame}

\begin{frame}\frametitle{Preliminary Results}\framesubtitle{\color{white}{Iceland Volcano (Eyjafjallaj\"{o}kull) Eruption}}
%\graphicspath{{/Users/psingla/Documents/PuneetDocs/Publications/Conferences/ACC/ACC10/stochForcingLin/Figures/}}

\begin{center}
\begin{figure}
\centering
\only<1>{\includegraphics[width=4in]{../figures/ash_ret_goesr_ash_top_height_in_geocatL2_Meteosat_9_2010106_1200.pdf}
\caption{SEVIRI Data: Ash Top Height (16th April, 2010)}}
\only<2>{\includegraphics[width=4in]{../figures/maxheight_of_concentration_excee-in-pcq-13CCUniformRe-height4M.pdf}
\caption{PCQ Runs: Ash Top Height (Mean) (16th April, 2010)}}
\only<3>{\includegraphics[width=4in]{../figures/maxheight_stddev-in-pcq-13CCUniformRe-height4M.pdf}
\caption{PCQ Runs: Ash Top Height (Std. Dev.) (16th April, 2010)}}
\end{figure}
\end{center}

\end{frame}

\begin{frame}\frametitle{Preliminary Results}\framesubtitle{\color{white}{Iceland Volcano (Eyjafjallaj\"{o}kull) Eruption}}

\begin{center}
\begin{figure}
\centering
\only<1>{\includegraphics[width=3.5in]{../figures/Top_Height_sat_15_16Apr.jpg}
\caption{SEVIRI Data: Ash Top Height (15th \&16th April, 2010)}}
\only<2>{\includegraphics[width=3.5in]{../figures/Top_Height_15_16Apr_oversamples.jpg}
\caption{PCQ Runs: Ash Top Height  (15th \& 16th April, 2010)}}
\end{figure}
\end{center}

\end{frame}


%\begin{frame}\frametitle{Preliminary Results}\framesubtitle{\color{white}{Iceland Volcano (Eyjafjallaj\"{o}kull) Eruption}}
%\begin{table}[htb]\footnotesize
%\begin{center}
%\caption{\footnotesize {Comparison of footprint forecast for April 16 (mean + $3\times$ standard deviation) and satellite data using $9^4$ and $13^4$ points of Clenshaw Curtis quadrature rule based PCQ. }
%%Dice=(Area of intersection)/(Area of union); PCQ given Sat=(Area of Intersection)/(Area of Satellite); Sat given PCQ=(Area of Intersection)/(Area of PCQ) 
%}\label{tab2}
%\begin{tabular}{|p{0.6in}|p{0.6in}|p{0.6in}|p{0.6in}|p{0.6in}|}
%\hline \hline
%Metric of Footprint & 0$<$h$<$5km $9^4$ points& 0$<$h$<$5km $13^4$ points & 5$<$h$<$10km $9^4$ points & 5$<$h$<$10km $13^4$ points \\
%\hline \hline
%Dice & 0.7367 & 0.7389& 0.9469 & 0.9479  \\ \hline
%\alert{PCQ given Sat} & \alert{0.9870} & \alert{0.9873} & \alert{0.9905} & \alert{0.9905}  \\ \hline
%\color{blue}{Sat given PCQ} & \color{blue}{0.7714}& \color{blue}{0.7460} & \color{blue}{0.9556} & \color{blue}{0.9565}  \\\hline
%\hline
%\end{tabular}
%\end{center}\vspace{-0.2in}
%\end{table}
%\scriptsize
%\begin{center}
%\begin{itemize}
%\item \textit{Dice} = $\frac{area(Sat\cap PCQ)}{area(Sat\cup PCQ)} $ is a measure of similarity of two images. 
%\item \alert{\textit{PCQ given Sat} = $\frac{area(Sat\cap PCQ)}{area(Sat)}$ is a measure of \textit{(1-false negatives)}.}
%\item {\color{blue}{\textit{Sat given PCQ} = $\frac{area(Sat\cap PCQ)}{area(PCQ)} $ is a measure of \textit{(1-false positives)}. }}
%\end{itemize}
%\end{center}
%\end{frame}
%
%\begin{frame}\frametitle{Preliminary Results}\framesubtitle{\color{white}{Iceland Volcano (Eyjafjallaj\"{o}kull) Eruption}}
%\begin{table}[htb]\footnotesize
%\begin{center}
%\caption{\footnotesize {Comparison of footprint forecast for April 15 \& 16  and satellite data using $9^4$ and $13^4$ points of Clenshaw Curtis quadrature rule based PCQ. }
%%Dice=(Area of intersection)/(Area of union); PCQ given Sat=(Area of Intersection)/(Area of Satellite); Sat given PCQ=(Area of Intersection)/(Area of PCQ) 
%}\label{tab2}
%\begin{tabular}{|p{0.6in}|p{0.6in}|p{0.6in}|p{0.6in}|p{0.6in}|}
%\hline \hline
%Metric of Footprint & 0$<$h$<$5km $9^4$ points& 0$<$h$<$5km $13^4$ points & 5$<$h$<$10km $9^4$ points & 5$<$h$<$10km $13^4$ points \\
%\hline \hline
%Dice & 0.6801 & 0.3926& 0.6905 & 0.7070  \\ \hline
%\alert{PCQ given Sat} & \alert{0.7804} & \alert{0.6879} & \alert{0.8679} & \alert{0.8655}  \\ \hline
%\color{blue}{Sat given PCQ} & \color{blue}{0.4777}& \color{blue}{0.8516} & \color{blue}{0.7716} & \color{blue}{0.7943}  \\\hline
%\hline
%\end{tabular}
%\end{center}\vspace{-0.2in}
%\end{table}
%\scriptsize
%\begin{center}
%\begin{itemize}
%\item \textit{Dice} = $\frac{area(Sat\cap PCQ)}{area(Sat\cup PCQ)} $ is a measure of similarity of two images. 
%\item \alert{\textit{PCQ given Sat} = $\frac{area(Sat\cap PCQ)}{area(Sat)}$ is a measure of \textit{(1-false negatives)}.}
%\item {\color{blue}{\textit{Sat given PCQ} = $\frac{area(Sat\cap PCQ)}{area(PCQ)} $ is a measure of \textit{(1-false positives)}. }}
%\end{itemize}
%\end{center}
%\end{frame}


\section{Concluding Remarks}

\begin{frame}\frametitle{Concluding Remarks}\framesubtitle{\color{white}{What needs to be done}}
\transboxin

\begin{itemize}
\item<1->  Efficient hybrid Bayesian approach is developed for the accurate determination of uncertainty propagation in linear dynamic models with
\begin{itemize}
\item<1-> \alert{\textit{Parametric, initial condition uncertainties, and driven by additive white Gaussian noise (AWGN) process.}}
\item<1-> Can be extended for nonlinear systems using {\color{blue}{\textit{Kolmogorov equation}.}}
\end{itemize}
\item<2-> The uncertainty due to stochastic forcing is propagated using {\color{blue}{mean and covariance propagation equations}} and that due to \textit{uncertain model parameters using polynomial chaos.}
\begin{itemize}
\item<2-> The moment propagation equations are \alert{exact only for white Gaussian stochastic forcing} in linear dynamic models, the polynomial chaos approach can be used for {\color{blue}{any probability distribution of model parameters.}}
\end{itemize}
\item<3->  Proposed approach is \alert{less computationally demanding} than the standard Monte Carlo methods.
\end{itemize}
\end{frame}

\begin{frame}\frametitle{Concluding Remarks}\framesubtitle{\color{white}{What needs to be done}}
\transboxin
\begin{itemize}
\item<1->  We were able to predict ash footprint with good confidence even though source uncertainty is large.
\begin{itemize}
\item<1-> {\color{blue}{\textit{False positives are large.}}}
\item<1-> Predicted average mass loading is of \textit{similar magnitude} assuming a 1-km thick down wind plume as suggested by CALIPSO data.
\item<1-> \alert{Convergence in concentration value is slow.}
\end{itemize}
\item<2-> The PCQ approach uses \texttt{BENT} and \texttt{PUFF} as black-box models
\begin{itemize}
\item<2->  \alert{Any other models} for ash dispersion can be included in the uncertainty analysis.
\end{itemize}
\item<3-> Effect of uncertainty in wind data still needs to be analyzed.
\begin{itemize}
\item<3->{\color{blue}{Uncertainty Marriage}}
\end{itemize}
\item<4->  \alert{Question arise:} \textit{Can we estimate the distribution of source parameters using Satellite imagery or other sensory data? }
\begin{itemize}
\item<4-> \alert{Likelihood functions:} validating the sensor data.
\item<4-> \textit{uniqueness of the parameters.}
\end{itemize}
\end{itemize}
\end{frame}


\end{document}
